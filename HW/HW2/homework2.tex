\documentclass[]{article}
\usepackage{lmodern}
\usepackage{amssymb,amsmath}
\usepackage{ifxetex,ifluatex}
\usepackage{fixltx2e} % provides \textsubscript
\ifnum 0\ifxetex 1\fi\ifluatex 1\fi=0 % if pdftex
  \usepackage[T1]{fontenc}
  \usepackage[utf8]{inputenc}
\else % if luatex or xelatex
  \ifxetex
    \usepackage{mathspec}
  \else
    \usepackage{fontspec}
  \fi
  \defaultfontfeatures{Ligatures=TeX,Scale=MatchLowercase}
\fi
% use upquote if available, for straight quotes in verbatim environments
\IfFileExists{upquote.sty}{\usepackage{upquote}}{}
% use microtype if available
\IfFileExists{microtype.sty}{%
\usepackage{microtype}
\UseMicrotypeSet[protrusion]{basicmath} % disable protrusion for tt fonts
}{}
\usepackage[margin=1in]{geometry}
\usepackage{hyperref}
\hypersetup{unicode=true,
            pdftitle={CSCI E-106:Assignment 2 Submisison},
            pdfauthor={Gurdeep Singh},
            pdfborder={0 0 0},
            breaklinks=true}
\urlstyle{same}  % don't use monospace font for urls
\usepackage{graphicx,grffile}
\makeatletter
\def\maxwidth{\ifdim\Gin@nat@width>\linewidth\linewidth\else\Gin@nat@width\fi}
\def\maxheight{\ifdim\Gin@nat@height>\textheight\textheight\else\Gin@nat@height\fi}
\makeatother
% Scale images if necessary, so that they will not overflow the page
% margins by default, and it is still possible to overwrite the defaults
% using explicit options in \includegraphics[width, height, ...]{}
\setkeys{Gin}{width=\maxwidth,height=\maxheight,keepaspectratio}
\IfFileExists{parskip.sty}{%
\usepackage{parskip}
}{% else
\setlength{\parindent}{0pt}
\setlength{\parskip}{6pt plus 2pt minus 1pt}
}
\setlength{\emergencystretch}{3em}  % prevent overfull lines
\providecommand{\tightlist}{%
  \setlength{\itemsep}{0pt}\setlength{\parskip}{0pt}}
\setcounter{secnumdepth}{0}
% Redefines (sub)paragraphs to behave more like sections
\ifx\paragraph\undefined\else
\let\oldparagraph\paragraph
\renewcommand{\paragraph}[1]{\oldparagraph{#1}\mbox{}}
\fi
\ifx\subparagraph\undefined\else
\let\oldsubparagraph\subparagraph
\renewcommand{\subparagraph}[1]{\oldsubparagraph{#1}\mbox{}}
\fi

%%% Use protect on footnotes to avoid problems with footnotes in titles
\let\rmarkdownfootnote\footnote%
\def\footnote{\protect\rmarkdownfootnote}

%%% Change title format to be more compact
\usepackage{titling}

% Create subtitle command for use in maketitle
\providecommand{\subtitle}[1]{
  \posttitle{
    \begin{center}\large#1\end{center}
    }
}

\setlength{\droptitle}{-2em}

  \title{CSCI E-106:Assignment 2 Submisison}
    \pretitle{\vspace{\droptitle}\centering\huge}
  \posttitle{\par}
    \author{Gurdeep Singh}
    \preauthor{\centering\large\emph}
  \postauthor{\par}
    \date{}
    \predate{}\postdate{}
  

\begin{document}
\maketitle

\hypertarget{due-date-september-21-2020-at-720-pm-est}{%
\subsubsection{Due Date: September 21, 2020 at 7:20 pm
EST}\label{due-date-september-21-2020-at-720-pm-est}}

\hypertarget{instructions}{%
\subsubsection{Instructions}\label{instructions}}

Students should submit their reports on Canvas. The report needs to
clearly state what question is being solved, step-by-step walk-through
solutions, and final answers clearly indicated. Please solve by hand
where appropriate.

Please submit two files: (1) a R Markdown file (.Rmd extension) and (2)
a PDF document, word, or html generated using knitr for the .Rmd file
submitted in (1) where appropriate. Please, use RStudio Cloud for your
solutions.

\begin{center}\rule{0.5\linewidth}{0.5pt}\end{center}

\hypertarget{problem-1}{%
\subsection{Problem 1}\label{problem-1}}

Refer to the regression model \(Y_{i} = \beta_{0} + \epsilon_{i}\).
(25pts)

a-) Derive the least squares estimator of \(\beta_{0}\) for this
model.(10pts)

b-) Prove that the least squares estimator of \(\beta_{0}\) is
unbiased.(5pts)

c-) Prove that the sum of the Y observations is the same as the sum of
the fitted values.(5pts)

d-) Prove that the sum of the residuals weighted by the fitted values is
zero.(5pts)

\hypertarget{problem-2}{%
\subsection{Problem 2}\label{problem-2}}

Refer to the Grade point average Data. The director of admissions of a
small college selected 120 students at random from the new freshman
class in a study to determine whether a student's grade point average
(GPA) at the end of the freshman year (Y) can be predicted from the ACT
test score (X). (30 points, each part is 5 points)

a-) Obtain a 99 percent confidence interval for \(\beta_{1}\). Interpret
your confidence interval. Does it include zero? Why might the director
of admissions be interested in whether the confidence interval includes
zero?

b-) Test, using the test statistic \(t^{*}\), whether or not a linear
association exists between student's ACT score (X) and GPA at the end of
the freshman year (Y). Use a level of significance of \(\alpha=0.01\).
State the alternatives, decision rule, and conclusion.

c-) What is the P-value of your test in part (b)? How does it support
the conclusion reached in part (b)?

d-)Obtain a 95 percent interval estimate of the mean freshman GPA for
students whose ACT test score is 28. Interpret your confidence interval.

e-) Mary Jones obtained a score of 28 on the entrance test. Predict her
freshman GPA-using a 95 percent prediction interval. Interpret your
prediction interval.

f-) Is the prediction interval in part (e) wider than the confidence
interval in part (d)? Should it be?

g-) Determine the boundary values of the 95 percent confidence band for
the regression line when \(X_{h}\) = 28. Is your-confidence band wider
at this point than the confidence interval in part (d)? Should it be?

\hypertarget{problem-3}{%
\subsection{Problem 3}\label{problem-3}}

Refer to the Crime rate data. A criminologist studying the relationship
between level of education-and crime rate in medium-sized U.S. counties
collected the following data for a random sample of 84 counties; X is
the percentage of individuals in the county having at least a
high-school diploma, and Y is the crime rate (crimes reported per
100,000 residents) last year. (45 points, each part is 5 points)

a-)Obtain the estimated regression function. Plot the estimated
regression function and the data. Does the linear regression function
appear to give a good fit here? Discuss.

b-) Test whether or not there is a linear association between crime rate
and percentage of high school graduates, using a t test with
\(\alpha =0.01\). State the alternatives, decision rule, and conclusion.
What is the P-value of the test?

c-) Estimate \(\beta_{1}\), with a 99 percent confidence interval.
Interpret your interval estimate.

d-) Set up the ANOVA table.

e-) Carry out the test in part a by means of the F test. Show the
numerical equivalence of the two test statistics and decision rules. Is
the P-value for the F test the same as that for the t test?

f-) By how much is the total variation in crime rate reduced when
percentage of high school graduates is introduced into the analysis? Is
this a relatively large or small reduction?

g-) State the full and reduced models.

h-) Obtain (1) SSE(F), (2) SSE(R), (3) dfF. (4) dfR, (5) test statistic
F* for the general linear test, (6) decision rule.

i-)Are the test statistic F* and the decision rule for the general
linear test numerically equivalent to those in part a?


\end{document}
